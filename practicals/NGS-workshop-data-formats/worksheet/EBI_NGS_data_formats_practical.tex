\documentclass{beamer}
\setbeameroption{hide notes}

%\documentclass[handout]{beamer}
%\setbeameroption{show notes}

%\documentclass{beamer}
%\usepackage{pgfpages}
%\setbeameroption{show notes on second screen}

%\documentclass[a4paper]{article}
%\usepackage{beamerarticle}

% Based on beamer solution template "generic-ornate-15min-45min.en.tex" by Till Tantau. 

\mode<presentation>
{
  \usetheme{Berkeley}
%  \usetheme{Berlin}
%  \usetheme{Frankfurt}
%  \usetheme{Warsaw}
%  \usetheme{Ilmenau}

%  \usecolortheme{seahorse}
%  \usecolortheme{dolphin}
%  \usecolortheme{whale}

%  \usecolortheme{rose}
%  \usecolortheme{orchid}

  \usefonttheme{default}

  \setbeamercovered{transparent}
  
}


\usepackage[english]{babel}

\usepackage[latin1]{inputenc}

%\usepackage{times}
%\usepackage[T1]{fontenc}
\usepackage{lmodern}
\usepackage[T1]{fontenc}



% support for \includesvg
\newcommand{\executeiffilenewer}[3]{%
\ifnum\pdfstrcmp{\pdffilemoddate{#1}}%
{\pdffilemoddate{#2}}>0%
{\immediate\write18{#3}}\fi%
} 
\newcommand{\includesvg}[1]{%
\executeiffilenewer{#1.svg}{#1.pdf}%
{inkscape -z -D --file=#1.svg %
--export-pdf=#1.pdf --export-dpi=72 --export-latex}%
\input{#1.pdf_tex}%
}



%\usepackage{caption}
%\usepackage{listings}
% setup listings captions
%\DeclareCaptionFont{white}{ \color{white} }
%\DeclareCaptionFormat{listing}{
%  \colorbox[cmyk]{0.43, 0.35, 0.35,0.01 }{
%    \parbox{\textwidth}{\hspace{15pt}#1#2#3}
%  }
%}
%\captionsetup[lstlisting]{ format=listing, labelfont=white, textfont=white, singlelinecheck=false, margin=0pt, font={bf,footnotesize} }

\usepackage[listings,theorems]{tcolorbox}

% custom listing styles
\lstdefinestyle{custombash}{
  belowcaptionskip=1\baselineskip,
  breaklines=true,
%  frame=L,
%  xleftmargin=\parindent,
  language=sh,
  showstringspaces=true,
  basicstyle=\tiny\ttfamily,
  keywordstyle=\bfseries\color{green!40!black},
  commentstyle=\itshape\color{purple!40!black},
  identifierstyle=\color{blue},
  stringstyle=\color{orange},
}

\lstdefinestyle{customsam}{
  belowcaptionskip=1\baselineskip,
  breaklines=true,
%  frame=L,
%  xleftmargin=\parindent,
%  language=SAM,
  showstringspaces=true,
  basicstyle=\tiny\ttfamily,
  keywordstyle=\bfseries\color{green!40!black},
  commentstyle=\itshape\color{purple!40!black},
  identifierstyle=\color{blue},
  stringstyle=\color{orange},
}


\usepackage{path}

\newcommand{\coursedate} {4th April 2016}

%%%%%%%%%%%%%%%%%%%%%%%%%%%%%%%%%%%%%%%%
% Title, Author, etc
%%%%%%%%%%%%%%%%%%%%%%%%%%%%%%%%%%%%%%%%
\title[EBI NGS Workshop]{EBI Next Generation Sequencing Workshop}
\subtitle{NGS Data Formats, \coursedate}

\author[]%[J.~C.~Randall]
{Joshua~C.~Randall}

\institute[Wellcome Trust Sanger Institute] 
{
  Senior Scientific Manager\\
  Human Genetics Informatics\\
  Wellcome Trust Sanger Institute
}

\date%[Short Occasion] % (optional)
{\coursedate}

%\subject{NGS resequencing, assembly, and variant calling} 
% This is only inserted into the PDF information catalog. Can be left out. 

%\pgfdeclareimage[height=0.5cm]{sanger-logo}{Sang_Log_Lge_2col}
%\logo{\pgfuseimage{sanger-logo}}
\pgfdeclareimage[height=0.5cm, width=0.97cm]{hgi-logo}{HGI-ltblue-nostroke}
\logo{\pgfuseimage{hgi-logo}}


% Delete this, if you do not want the table of contents to pop up at
% the beginning of each subsection:
\AtBeginSubsection[]
{
  \begin{frame}<beamer>{Outline}
    \tableofcontents[currentsection,currentsubsection]
  \end{frame}
}


% If you wish to uncover everything in a step-wise fashion, uncomment
% the following command: 

%\beamerdefaultoverlayspecification{<+->}


\begin{document}


%%%%%%%%%%%%%%%%%%%%%%%%%%%%%%%%%%%%%%%%
% Title 
%%%%%%%%%%%%%%%%%%%%%%%%%%%%%%%%%%%%%%%%
\begin{frame}
  \titlepage
\end{frame}


%%%%%%%%%%%%%%%%%%%%%%%%%%%%%%%%%%%%%%%%
% Outline
%%%%%%%%%%%%%%%%%%%%%%%%%%%%%%%%%%%%%%%%
\begin{frame}{Outline}
  \tableofcontents
  % You might wish to add the option [pausesections]
\end{frame}


% Since this a solution template for a generic talk, very little can
% be said about how it should be structured. However, the talk length
% of between 15min and 45min and the theme suggest that you stick to
% the following rules:  
% 
% - Exactly two or three sections (other than the summary).
% - At *most* three subsections per section.
% - Talk about 30s to 2min per frame. So there should be between about
%   15 and 30 frames, all told.

%\section{Introduction}
%
%\subsection[Short First Subsection Name]{First Subsection Name}
%
%\begin{frame}{Make Titles Informative. Use Uppercase Letters.}{Subtitles are optional.}
%  % - A title should summarize the slide in an understandable fashion
%  %   for anyone how does not follow everything on the slide itself.
%
%  \begin{itemize}
%  \item
%    Use \texttt{itemize} a lot.
%  \item
%    Use very short sentences or short phrases.
%  \end{itemize}
%\end{frame}
%
%\begin{frame}{Make Titles Informative.}
%
%  You can create overlays\dots
%  \begin{itemize}
%  \item using the \texttt{pause} command:
%    \begin{itemize}
%    \item
%      First item.
%      \pause
%    \item    
%      Second item.
%    \end{itemize}
%  \item
%    using overlay specifications:
%    \begin{itemize}
%    \item<3->
%      First item.
%    \item<4->
%      Second item.
%    \end{itemize}
%  \item
%    using the general \texttt{uncover} command:
%    \begin{itemize}
%      \uncover<5->{\item
%        First item.}
%      \uncover<6->{\item
%        Second item.}
%    \end{itemize}
%  \end{itemize}
%\end{frame}
%
%
%\subsection{Second Subsection}
%
%\begin{frame}{Make Titles Informative.}
%\end{frame}
%
%\begin{frame}{Make Titles Informative.}
%\end{frame}
%\section{Lecture}


%\section{Practical}


%%%%%%%%%%%%%%%%%%%%%%%%%%%%%%%%%%%%%%%%
% Exercises
%%%%%%%%%%%%%%%%%%%%%%%%%%%%%%%%%%%%%%%%
\section{Exercises}

\begin{frame}[fragile]{Exercises}
%\begin{itemize}
%\item 
%\end{itemize}
\begin{tcblisting}{title={Prepare Exercise Data}, listing only, listing options={style=custombash}}
# expand archive
tar zxvf EBI_NGS_data_formats_lab.tgz
# change into lab directory
cd EBI_NGS_data_formats_lab
\end{tcblisting}
\end{frame}

% # wget ftp://ftp.1000genomes.ebi.ac.uk/vol1/ftp/data/NA12878/alignment/NA12878.chrom20.ILLUMINA.bwa.CEU.low_coverage.20121211.bam
% # samtools view -s 314159.1 -b NA12878.chrom20.ILLUMINA.bwa.CEU.low_coverage.20121211.bam > NA12878chr20.bam
% # md5sum NA12878chr20.bam
% 5d3a4c4a25950db62d40bbde7e408b3d  NA12878chr20.bam

% # wget ftp://ftp.1000genomes.ebi.ac.uk/vol1/ftp/release/20130502/ALL.chr22.phase3_shapeit2_mvncall_integrated_v5.20130502.genotypes.vcf.gz
% # bcftools view -s "NA12003,NA12004,NA12005,NA12006,NA12043,NA12044,NA12045,NA12046,NA12058,NA12144,NA12154,NA12155,NA12156,NA12234,NA12249,NA12272,NA12273,NA12275,NA12282,NA12283,NA12286,NA12287,NA12340,NA12341,NA12342,NA12347,NA12348,NA12383,NA12399,NA12400,NA12413,NA12414,NA12489,NA12546,NA12716,NA12717,NA12718,NA12748,NA12749,NA12750,NA12751,NA12760,NA12761,NA12762,NA12763,NA12775,NA12776,NA12777,NA12778,NA12812,NA12813,NA12814,NA12815,NA12827,NA12828,NA12829,NA12830,NA12842,NA12843,NA12872,NA12873,NA12874,NA12878,NA12889,NA12890" -O z -o ALL-NA12-samples.chr22.phase3_shapeit2_mvncall_integrated_v5.20130502.genotypes.vcf.gz  ALL.chr22.phase3_shapeit2_mvncall_integrated_v5.20130502.genotypes.vcf.gz
% # md5sum ALL-NA12-samples.chr22.phase3_shapeit2_mvncall_integrated_v5.20130502.genotypes.vcf.gz
% 0b83040d3352666b81d1e29e167f244d  ALL-NA12-samples.chr22.phase3_shapeit2_mvncall_integrated_v5.20130502.genotypes.vcf.gz
% # mv ALL-NA12-samples.chr22.phase3_shapeit2_mvncall_integrated_v5.20130502.genotypes.vcf.gz NA12xxxchr22.vcf.gz

\subsection{Data Formats: SAM/BAM}
\begin{frame}[fragile]{Exercises}
\framesubtitle{SAM/BAM Format}
\begin{itemize}
\item SAM is the accepted standard format for storing NGS reads, base qualities, metadata, and alignments to a reference genome
\item Review the `SAMv1' specification from \url{http://samtools.github.io/hts-specs/} and consider the following header line: 
\end{itemize}
\begin{tcblisting}{listing only, listing options={style=customsam}}
@RG	ID:ERR001711	PL:ILLUMINA	LB:g1k-sc-NA12878-CEU-1	PI:200	DS:SRP000032	SM:NA12878	CN:SC
\end{tcblisting}
\begin{itemize}
\item What does RG stand for?
\item What is the sequencing platform?
\item What is the expected fragment insert size?
\end{itemize}
\end{frame}



\begin{frame}[fragile]{Exercises}
\framesubtitle{SAM/BAM Format}
\begin{itemize}
\item Change to the \path !Exercise-formats! directory
\item Use `\texttt{samtools view}' to convert BAM to SAM and print the header
\end{itemize}
\begin{tcblisting}{listing only, listing options={style=custombash}}
samtools view -H NA12878chr20.bam
\end{tcblisting}
\begin{itemize}
\item What version of the human reference assembly was used to perform the alignments?
\item How many lanes are in this BAM file?
\item What version of bwa was used to align the reads?
\item What other programs were used to create this BAM file?
\end{itemize}
\end{frame}


\begin{frame}[fragile]{Exercises}
\framesubtitle{SAM/BAM Format}
\begin{itemize}
\item You can also use `\texttt{samtools view}' to view or extract regions of a BAM file.
\end{itemize}
\begin{tcblisting}{listing only, listing options={style=custombash}}
samtools view -h NA12878chr20.bam | less -S
\end{tcblisting}
\begin{itemize}
\item What is the name of the first read?
\item At what position does the alignment of the read start?
\item What is the mapping quality of the first read?
\end{itemize}
\end{frame}


\subsection{Alignment QC stats}
\begin{frame}[fragile]{Exercises}
\framesubtitle{Alignment QC stats}
\begin{itemize}
\item {Run samtools flagstat}
         \begin{tcolorbox}[fontupper=\scriptsize]
         We can use samtools to report basic statistics about the alignments in the BAM file we used in the last exercise. Run `\texttt{samtools flagstat}' on \path !NA12878chr20.bam!.
         \end{tcolorbox}
\item Looking at the flagstat output, answer the following questions:
	\begin{itemize}
	\item What is the total number of reads?
	\item What proportion of the reads were mapped?
	\item How many reads were paired correctly/properly?
	\item How many reads mapped to a different chromosome with map quality >=5?
	\end{itemize}
\end{itemize}
\end{frame}

\subsection{Alignment diagnostic plots}
\begin{frame}[fragile]{Exercises}
\framesubtitle{Alignment diagnostic plots}
\begin{itemize}
\item Run samtools stats
         \begin{tcolorbox}[fontupper=\scriptsize]
         Run `\texttt{samtools stats}' on \path !NA12878chr20.bam! and redirect \\
         output to \path !NA12878chr20.bam.stats!
         \end{tcolorbox}
\item Generate bamstats plots
         \begin{tcolorbox}[fontupper=\scriptsize]
         `\texttt{plot-bamstats -p bamstats/ NA12878chr20.bam.stats}'
         \end{tcolorbox}
\item View png plots (e.g. using `\texttt{mirage bamstats/*.png}')
\item Looking at the plots, answer the following questions:
	\begin{itemize}
	\item Does GC content vary between the fwd and rev reads?
	\item What is the median insert size?
	\item Which has higher average quality, fwd or rev reads?
	\item What can you tell about reverse read quality distribution?
	\end{itemize}
\end{itemize}
\end{frame}

%
%\subsection{BAM Visualisation}
%\begin{frame}[fragile]{Exercises}
%\framesubtitle{BAM Visualisation}
%IGV (\url{http://www.broadinstitute.org/igv/}) is a Java visualisation tool for looking at alignments of reads onto a reference genome from BAM files
%\begin{itemize}
%\item Launch IGV (using the Desktop shortcut)
%\item Load the reference genome located in \path !ref! by going to the `\texttt{Genome}' menu and selecting `\texttt{Load Genome From File...}'
%\item Load the \path !library.markdup.bam! BAM file by going to the `\texttt{File}' menu and selecting `\texttt{Load From File...}'
%\item Go to chromosome IV and position $764,294$ using the top navigation bar
%\item What is the reference base at this position?
%\item Do the reads agree with the reference base?
%\item How about at position $764,292$?
%\end{itemize}
%\end{frame}


\subsection{Data Formats: VCF}
\begin{frame}[fragile]{Exercises}
\framesubtitle{Variant Call Format}
\begin{itemize}
\item Variant Call Format (VCF) is a generic format for storing DNA polymorphism data (such as SNPs, insertions, deletions, and structural variants) alongside rich annotations.
\item VCF is text-based and can be read easily (e.g. using `\texttt{less}' for uncompressed files or `\texttt{zless}' for gzipped files). 
\item Headers describe the meaning of tag values
\end{itemize}
\begin{tcblisting}{listing only, listing options={style=custombash}}
zless -S NA12xxxchr22.vcf.gz
\end{tcblisting}
\begin{itemize}
\item What do the values of the DP INFO tag correspond to?
\item What sort of event is at position 22:16056483?
\item What is the total depth at position 22:16056801?
\item What is the allele frequency at position 22:16057417?
\end{itemize}
\end{frame}


\subsection{Variant diagnostic plots}
\begin{frame}[fragile]{Exercises}
\framesubtitle{Variant diagnostic plots}
\begin{itemize}
\item Run bcftools stats
         \begin{tcolorbox}[fontupper=\scriptsize]
         Run `\texttt{bcftools stats}' on \path !NA12xxxchr22.vcf.gz! and redirect \\
         output to \path !NA12xxxchr22.vcf.gz.stats!
         \end{tcolorbox}
\item Generate vcfstats plots
         \begin{tcolorbox}[fontupper=\scriptsize]
         `\texttt{plot-vcfstats -p vcfstats/ NA12xxxchr22.vcf.gz.stats}'
         \end{tcolorbox}
\item View vcfstats/summary.pdf
\item Looking at the plots, answer the following questions:
	\begin{itemize}
	\item What is the overall transition/transversion ratio?
	\item What substitution type is the most common?
	\end{itemize}
\end{itemize}
\end{frame}



%%%%%%%%%%%%%%%%%%%%%%%%%%%%%%%%%%%%%%%%
% Summary
%%%%%%%%%%%%%%%%%%%%%%%%%%%%%%%%%%%%%%%%
\section*{Summary}
% n.b. use \section* to keep out of TOC

\begin{frame}{Summary}

  % Keep the summary *very short*.
  \begin{itemize}
  \item
    Most important file formats to understand: SAM, VCF
    \begin{itemize}
    \item SAM specification \\
    (\url{http://samtools.github.io/hts-specs/SAMv1.pdf})
    \item VCF specification \\
    (\url{http://samtools.github.io/hts-specs/VCFv4.2.pdf})
    \end{itemize}
  \item
    Can convert between formats, filter, merge, etc using: 
    \begin{itemize}
    \item samtools for SAM/BAM/CRAM (\url{http://htslib.org/})
    \item bcftools for BCF/VCF (\url{http://htslib.org/})
%    \item IGV (\url{http://broadinstitute.org/igv/})
    \end{itemize}
%  \item
%    The \alert{second main message} of your talk in one or two lines.
%  \item
%    Perhaps a \alert{third message}, but not more than that.
  \end{itemize}
  
  % The following outlook is optional.
%  \vskip0pt plus.5fill
%  \begin{itemize}
%  \item
%    Outlook
%    \begin{itemize}
%    \item
%      Something you haven't solved.
%    \item
%      Something else you haven't solved.
%    \end{itemize}
%  \end{itemize}
\end{frame}


%%%%%%%%%%%%%%%%%%%%%%%%%%%%%%%%%%%%%%%%%
%% Installation
%%%%%%%%%%%%%%%%%%%%%%%%%%%%%%%%%%%%%%%%%
%\section*{Appendix: Software Installation}
%
%
%%\subsection{curl}
%%\begin{frame}[fragile]{Installation}
%%\framesubtitle{curl}
%%A command-line utility and library for downloading files via http(s).
%%\begin{tcblisting}{title={Install curl}, listing only, listing options={style=custombash}}
%%# download the source from the curl website
%%cd ~
%%wget http://git-core.googlecode.com/files/git-1.8.2.tar.gz
%%# unpack
%%tar zxvf git-1.8.2.tar.gz
%%# build 
%%cd git-1.8.2
%%./configure --prefix=$(echo ~/local)
%%make 
%%# install
%%make install
%%\end{tcblisting}
%%
%%Prerequisites: 
%%% sudo apt-get install gcc
%%gcc 
%%% sudo apt-get install make
%%make
%%\end{frame}
%
%
%%\subsection{git}
%%\begin{frame}[fragile]{Installation}
%%\framesubtitle{git}
%%\begin{tcblisting}{title={Install git}, listing only, listing options={style=custombash}}
%%# download the source from google code
%%cd ~
%%wget http://git-core.googlecode.com/files/git-1.8.2.tar.gz
%%# unpack
%%tar zxvf git-1.8.2.tar.gz
%%# build 
%%cd git-1.8.2
%%./configure --prefix=$(echo ~/local)
%%make 
%%# install
%%make install
%%\end{tcblisting}
%%
%%Prerequisites: 
%%% sudo apt-get install gcc
%%gcc 
%%% sudo apt-get install make
%%make
%%\end{frame}
%
%
%
%\subsection*{Environment and Prerequisites}
%\begin{frame}[fragile]{Appendix: Software Installation}
%\framesubtitle{Setup Environment}
%\begin{tcblisting}{title={Setup Environment}, listing only, listing options={style=custombash}}
%####################################################
%# set environment variables
%####################################################
%
%# add ~/local/bin to PATH
%export PATH=~/local/bin:$PATH
%
%# add ~/local/lib/perl5 to PERL5LIB
%export PERL5LIB=~/local/lib/perl5:$PERL5LIB
%
%\end{tcblisting}
%\end{frame}
%
%
%\begin{frame}[fragile]{Appendix: Software Installation}
%\framesubtitle{Install Prerequisitites}
%\begin{tcblisting}{title={Install Prerequisites}, listing only, listing options={style=custombash}}
%####################################################
%# install prerequisite packages 
%# (names may be specific to Ubuntu Linux >= 12.04)
%####################################################
%
%sudo apt-get update
%# these core utilities are probably already installed 
%sudo apt-get install bash coreutils make g++ gcc perl wget
%# java is also probably already installed
%sudo apt-get install ant default-jdk default-jre icedtea-netx 
%# these SCM tools may already be installed 
%sudo apt-get install git subversion 
%# development headers for samtools and bwa
%sudo apt-get install ncurses-dev zlib1g-dev 
%# gnuplot for plotting in plot-bamcheck
%sudo apt-get install gnuplot 
%
%####################################################
%# install useful utilities 
%####################################################
%
%# less for viewing text
%sudo apt-get install less 
%# mirage for viewing images
%sudo apt-get install mirage
%\end{tcblisting}
%\end{frame}
%
%
%\subsection*{bwa}
%\begin{frame}[fragile]{Appendix: Software Installation}
%\framesubtitle{bwa}
%\begin{tcblisting}{title={Install Development Version of bwa}, listing only, listing options={style=custombash}}
%# download the source from github
%cd ~
%git clone https://github.com/lh3/bwa.git
%
%# build 
%cd bwa
%make 
%
%# install
%install -s -D ./bwa ~/local/bin/bwa
%\end{tcblisting}
%
%Prerequisites: 
%% sudo apt-get install gcc
%gcc,
%% sudo apt-get install make
%make,
%% sudo apt-get install zlib1g-dev
%zlib 
%
%% may need CFLAGS=-msse2 on some machines
%\end{frame}
%
%
%\subsection*{samtools}
%\begin{frame}[fragile]{Appendix: Software Installation}
%\framesubtitle{samtools}
%\begin{tcblisting}{title={Install Development Version of samtools}, listing only, listing options={style=custombash}}
%# download the source from github
%cd ~
%git clone https://github.com/samtools/samtools.git
%
%# build 
%cd samtools
%make
%
%# install
%install -s -D ./samtools ~/local/bin/samtools
%install -s -D ./bcftools/bcftools ~/local/bin/bcftools
%install -s -D ./misc/bamcheck ~/local/bin/bamcheck
%install -D ./misc/plot-bamcheck ~/local/bin/plot-bamcheck
%\end{tcblisting}
%
%Prerequisites: 
%% sudo apt-get install gcc
%gcc,
%% sudo apt-get install make
%make,
%% sudo apt-get install zlib1g-dev
%zlib, 
%% sudo apt-get install ncurses-dev
%ncurses, 
%% sudo apt-get install perl
%perl, % (for plot-bamcheck), 
%% sudo apt-get install gnuplot
%gnuplot %(for plot-bamcheck)
%
%\end{frame}
%
%
%
%\subsection*{bgzip/tabix}
%\begin{frame}[fragile]{Appendix: Software Installation}
%\framesubtitle{bgzip/tabix}
%\begin{tcblisting}{title={Install Development Version of bgzip/tabix}, listing only, listing options={style=custombash}}
%# download the source from github
%cd ~
%git clone https://github.com/samtools/tabix.git
%
%# build 
%cd tabix
%make
%
%# install
%install -s -D ./bgzip ~/local/bin/bgzip
%install -s -D ./tabix ~/local/bin/tabix
%\end{tcblisting}
%
%Prerequisites: 
%% sudo apt-get install gcc
%gcc,
%% sudo apt-get install make
%make,
%% sudo apt-get install zlib1g-dev
%zlib, 
%% sudo apt-get install ncurses-dev
%ncurses, 
%% sudo apt-get install perl
%perl, % (for plot-bamcheck), 
%% sudo apt-get install gnuplot
%gnuplot %(for plot-bamcheck)
%
%\end{frame}
%
%
%\subsection*{vcftools}
%\begin{frame}[fragile]{Appendix: Software Installation}
%\framesubtitle{vcftools}
%%# download latest version of vcftools
%%wget -O vcftools_latest.tar.gz http://sourceforge.net/projects/vcftools/files/latest/download?source=files
%\begin{tcblisting}{title={Install Development Version of vcftools}, listing only, listing options={style=custombash}}
%# download the source from sourceforge
%cd ~
%svn checkout http://svn.code.sf.net/p/vcftools/code vcftools
%
%# build
%cd vcftools
%make
%
%# install scripts
%for file in $(ls ./bin); \
%  do install -D ./bin/${file} ~/local/bin/${file}; \
%done
%# install libraries
%for file in $(ls ./lib/perl5/site_perl); \
%  do install -D ./lib/perl5/site_perl/${file} ~/local/lib/perl5/${file}; \
%done
%# install vcftools binary
%install -s -D ./bin/vcftools ~/local/bin/vcftools
%\end{tcblisting}
%
%Prerequisites: 
%% sudo apt-get install g++
%g++
%% sudo apt-get install make
%make
%
%\end{frame}
%
%
%\subsection*{Picard}
%\begin{frame}[fragile]{Appendix: Software Installation}
%\framesubtitle{Picard}
%\begin{tcblisting}{title={Install Development Version of Picard}, listing only, listing options={style=custombash}}
%# download the source from sourceforge
%cd ~
%svn checkout http://svn.code.sf.net/p/picard/code/trunk picard
%
%# build
%cd picard
%ant -lib lib/ant package-commands
%
%# install
%mkdir -p ~/local/bin/
%install -D ./dist/*.jar ~/local/bin/
%\end{tcblisting}
%
%Prerequisites: 
%% sudo apt-get install default-jdk
%Java Development Kit (JDK), \\
%% sudo apt-get install java
%Java Runtime Environment (JRE), 
%% sudo apt-get install ant
%ant
%
%\end{frame}
%
%
%\subsection*{IGV}
%\begin{frame}[fragile]{Appendix: Software Installation}
%\framesubtitle{Install \& Launch IGV}
%\begin{tcblisting}{title={Install \& Launch IGV}, listing only, listing options={style=custombash}}
%# launch IGV using Java Web Start 
%# (also opens GUI and offers to install shortcut on Desktop)
%javaws http://www.broadinstitute.org/igv/projects/current/igv.jnlp
%\end{tcblisting}
%\begin{itemize}
%\item Select `Always trust content from this publisher'
%\item Press the `Run' button
%\item Press the `Allow' button to allow IGV to install a shortcut on the Desktop
%\end{itemize}
%Prerequisites:
%% sudo apt-get install java
%Java Runtime Environment (JRE), \\
%% sudo apt-get install icedtea-netx
%JNLP web start (javaws)
%\end{frame}
%





\end{document}

